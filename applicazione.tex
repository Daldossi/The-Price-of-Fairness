\documentclass{beamer}
%\usepackage[utf8]{inputenc}
%\usepackage[T1]{fontenc}
%\usepackage[italian]{babel}
\usetheme{Berlin}
\usefonttheme{serif}

\newtheorem{hp}{Ipotesi}
%\newenvironment<>{hp}[1][]{%
	%	\setbeamercolor{Ipotesi}{fg=white,bg=red!75!black}%
	%	\begin{example}#2[#1]}{\end{example}}
\newtheorem{prop}{Proposizione}

\usepackage{amsmath}
\usepackage{eurosym}
\usepackage{mathtools}
\usepackage{url}
\usepackage{hyperref}


%%%%%%%%%%%%%%%%%%%%%%%%%%%%%%%%%%%%%%%%%%%%%%%%%%%%%%%%%%%%%%%%%%%%%%%%%%%%%%%%%%%%%%%%%%%%%%%%%%%%%%%%%%%%%%
%Information to be included in the title page:
\title{The Price of Fairness\\Applicazione}
\author{Alice Daldossi}
\institute{Università degli Studi di Pavia}
\date

\begin{document}
	
	\frame{\titlepage}
	
\begin{frame}
	\frametitle{Indice}
	\tableofcontents
\end{frame}

\section{Problema}

\begin{frame}{Presentazione del problema}
\begin{block}{Problema}
	Un condominio da 6 appartamenti ha installato dei pannelli fotovoltaici che creano energia elettrica pari a 30 kWh al giorno. Questo totale viene normalmente suddiviso tra le 6 utenze in base alle quote di ciascuna. Se una famiglia va in vacanza, l'appartamento consuma meno, quindi c'è più energia a disposizione per le altre. Come distribuire questa energia in più?
\end{block}
\begin{block}{Risoluzione}
	 Per ogni famiglia si cerca la percentuale del rispettivo surplus (energia che non viene normalmente coperta dai fotovoltaici) che è coperta dal fotovoltaico aggiuntivo.
\end{block}
\end{frame}

\begin{frame}{Dati}
	$n=6$ giocatori\\
	Disponibilità giornaliera dei pannelli fotovoltaici: $30$ kWh\\
	Prezzo dell'energia: $0.277$ \euro/kWh
	\begin{tabular}{|l|l|l|l|l|}
		\hline
		\textbf{Presenze} & \textbf{Nomi} & \textbf{Copertura} & \textbf{Surplus} & \textbf{Fisso}\\
		\hline
		1 & Bianchi & 2,647058824 & 1,452941176 & 1,5 \\
		1 & Rossi & 3,235294118 & 4,164705882 & 1,5	\\
		1 & Verdi & 3,529411765 & 1,970588235 & 1,5	\\
		0 & Longo & 5,882352941 & 3,117647059 & 1,5	\\
		0 & Costa & 6,470588235 & 3,529411765 & 2 \\
		1 & Gatti & 8,235294118 & 5,764705882 & 2 \\
		\hline
	\end{tabular}
\end{frame}

\begin{frame}{Soluzione utilitaria}
%	\begin{equation}
%		\begin{split}
%			\underset{y}{\text{max}} & \qquad e^T u, \\
%			\text{t.c.} & \qquad u \in U.
%		\end{split}
%	\end{equation}
	\begin{center}
		\begin{tabular}{|l|l|l|}
		\hline
		\textbf{Nomi} & \textbf{Copertura aggiuntiva} & \textbf{Costo (\euro)}\\
		\hline
		Bianchi & $1.452941176$ & $0.00$ \\
		Rossi & $4.164705882$ & $0.00$ \\
		Verdi & $1.970588235$ & $0.00$ \\
		Gatti & $3.1176470600000004$ & $0.73$ \\
		\hline
	\end{tabular}
	\end{center}
	\begin{block}{Soluzione}
		\vspace{-0.5cm}
		\begin{equation}
			\begin{split}
				\text{SYSTEM}(U) = \sup \{e^T u | u \in U\} = \\
				=  100.0 + 100.0 + 100.0 + 54.08 = 354.08
			\end{split}
		\end{equation}
	\vspace{-0.3cm}
	\end{block}
	Link al codice: ...
\end{frame}

\section{Schemi di equità}

\begin{frame}{Equità proporzionale}
%	\begin{equation}
%		\begin{split}
%			\underset{y}{\text{max}} & \qquad \prod_{j=1}^n u_j \\
%			\text{t.c.} & \qquad u \in U.
%		\end{split}
%	\end{equation}
	\begin{block}{Soluzione}
		\vspace{-0.7cm}
		\begin{equation}
			\begin{split}
				& \mathcal{S}^{\text{PF}}(U) = \\
				& \text{FAIR}(U;\mathcal{S}^{\text{PF}}) = e^T \mathcal{S}^{\text{PF}}(U) = \\
				& \text{POF}(U;\mathcal{S}^{\text{PF}}) = \frac{\text{SYSTEM}(U) - \text{FAIR}(U;\mathcal{S}^{\text{PF}})}{\text{SYSTEM}(U)}
			\end{split}
		\end{equation}
	\vspace{-0.4cm}
	\end{block}
	Link al codice: ...
\end{frame}

\begin{frame}{Equità max-min}
%	... Problema ...
	\begin{block}{Soluzione}
		\vspace{-0.7cm}
		\begin{equation}
			\begin{split}
				& \mathcal{S}^{\text{MMF}}(U) = \\
				& \text{FAIR}(U;\mathcal{S}^{\text{MMF}}) = e^T \mathcal{S}^{\text{MMF}}(U) = \\
				& \text{POF}(U;\mathcal{S}^{\text{MMF}}) = \frac{\text{SYSTEM}(U) - \text{FAIR}(U;\mathcal{S}^{\text{MMF}})}{\text{SYSTEM}(U)}
			\end{split}
		\end{equation}
		\vspace{-0.4cm}
	\end{block}
	Link al codice: ...
\end{frame}

\begin{frame}{Limiti}
	\begin{block}{Teorema: Uguali massime utilità possibili}
		Abbiamo $ n = 6 > 2$ giocatori, $U \subset \mathbb{R}^6_+$ è compatto e convesso. Tutti i giocatori hanno la stessa massima utilità, che è maggiore di 0, allora
		\begin{equation}
			\begin{split}
				& \text{POF}(U;\mathcal{S}^{\text{PF}}) = \dots \le 1 - \frac{2\sqrt{n}-1}{n} = 0.3501700857389407, \\ 
				& \text{POF}(U;\mathcal{S}^{\text{MMF}}) = \dots \le 1 - \frac{4n}{(n+1)^2} = 0.5102040816326531. 
			\end{split}
		\end{equation}
	\end{block}
\end{frame}

\end{document}